%-*- coding: UTF-8 -*-

\documentclass[UTF8]{ctexart}
\usepackage{graphicx}
\usepackage{float}
\usepackage{color}
\usepackage{CJKpunct}
\usepackage{amsmath}
\usepackage{geometry}
\geometry{a4paper,centering,scale=0.8}
\usepackage[format=hang,font=small,textfont=it]{caption}
\usepackage[nottoc]{tocbibind}
\setCJKmainfont{STSongti-TC-Light}
\setromanfont{FandolSong}
\punctstyle{quanjiao} %使用全角标点	
\linespread{1.5} 

\begin{document}

\section*{長恨歌}

THE EVERLASTING REGRET

\section{貴妃受寵}

漢皇重色思傾國,御宇多年求不得。楊家有女初長成,養在深閨人未識。

天生麗質難自棄,一朝選在君王側。回眸一笑百媚生,六宮粉黛無顏色。

春寒賜浴華清池,溫泉水滑洗凝脂。侍兒扶起嬌無力,始是新承恩澤時。

雲鬢花顏金步搖,芙蓉帳暖度春宵。春宵苦短日高起,從此君王不早朝。

承歡侍宴無閒暇,春從春遊夜專夜。後宮佳麗三千人,三千寵愛在一身。

金屋妝成嬌侍夜,玉樓宴罷醉和春。姊妹弟兄皆列土,可憐光彩生門戶。

遂令天下父母心,不重生男重生女。驪宮高處入青雲,仙樂風飄處處聞。

緩歌慢舞凝絲竹,盡日君王看不足。漁陽鼙鼓動地來,驚破霓裳羽衣曲。

\section{馬嵬驚變}

九重城闕煙塵生,千乘萬騎西南行。翠華搖搖行復止,西出都門百餘里。

六軍不發無奈何,宛轉蛾眉馬前死。花鈿委地無人收,翠翹金雀玉搔頭。

君王掩面救不得,回看血淚相和流。黃埃散漫風蕭索,雲棧縈紆登劍閣。

峨嵋山下少人行,旌旗無光日色薄。蜀江水碧蜀山青,聖主朝朝暮暮情。

行宮見月傷心色,夜雨聞鈴腸斷聲。

\section{玄宗思舊}

天旋地轉迴龍馭,到此躊躇不能去。馬嵬坡下泥土中,不見玉顏空死處。

君臣相顧盡霑衣,東望都門信馬歸。歸來池苑皆依舊,太液芙蓉未央柳。

芙蓉如面柳如眉,對此如何不淚垂。春風桃李花開日,秋雨梧桐葉落時。

西宮南內多秋草,落葉滿階紅不掃。梨園弟子白髮新,椒房阿監青娥老。

夕殿螢飛思悄然,孤燈挑盡未成眠。遲遲鐘鼓初長夜,耿耿星河欲曙天。

鴛鴦瓦冷霜華重,翡翠衾寒誰與共。悠悠生死別經年,魂魄不曾來入夢。

\section{仙界尋妃}

臨邛道士鴻都客,能以精誠致魂魄。為感君王輾轉思,遂教方士殷勤覓。

排空馭氣奔如電,昇天入地求之遍。上窮碧落下黃泉,兩處茫茫皆不見。

忽聞海上有仙山,山在虛無縹緲間。樓閣玲瓏五雲起,其中綽約多仙子。

中有一人字太真,雪膚花貌參差是。金闕西廂叩玉扃,轉教小玉報雙成。

聞道漢家天子使,九華帳裏夢魂驚。攬衣推枕起徘徊,珠箔銀屏迤邐開。

雲髻半偏新睡覺,花冠不整下堂來。風吹仙袂飄颻舉,猶似霓裳羽衣舞。

玉容寂寞淚闌干,梨花一枝春帶雨。含情凝睇謝君王,一別音容兩渺茫。

昭陽殿裏恩愛絕,蓬萊宮中日月長。回頭下望人寰處,不見長安見塵霧。

唯將舊物表深情,鈿合金釵寄將去。釵留一股合一扇,釵擘黃金合分鈿。

但教心似金鈿堅,天上人間會相見。臨別殷勤重寄詞,詞中有誓兩心知。

七月七日長生殿,夜半無人私語時。在天願作比翼鳥,在地願為連理枝。

天長地久有時盡,此恨綿綿無絕期。

\end{document} 