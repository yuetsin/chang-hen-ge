\documentclass[UTF8]{ctexart}
\usepackage{graphicx}
\usepackage{float}
\usepackage{color}
\usepackage{CJKpunct}
\usepackage{amsmath}
\usepackage{geometry}
\geometry{a4paper,centering,scale=0.8}
\usepackage[format=hang,font=small,textfont=it]{caption}
\usepackage[nottoc]{tocbibind}
\setCJKmainfont{STSongti-SC-Light}
\punctstyle{quanjiao} %使用全角标点	
\linespread{1.5} 
\begin{document}

\section*{长恨歌}

THE EVERLASTING REGRET

\section{汉皇重色思倾国}

汉皇重色\footnote{爱好女色。倾国:绝色女子。汉代李延年对汉武帝唱了一首歌:“北方有佳人,绝世而独立。一顾倾人城,再顾倾人国。宁不知倾国与倾城,佳人难再得。”后来,“倾国倾城”就成为美女的代称。}思倾国,御宇\footnote{驾御宇内,即统治天下。汉贾谊《过秦论》:“振长策而御宇内”。}多年求不得。

杨家有女\footnote{蜀州司户杨玄琰,有女杨玉环,自幼由叔父杨玄珪抚养,十七岁(开元二十三年)被册封为玄宗之子寿王李瑁之妃。二十七岁被玄宗册封为贵妃。白居易此谓“养在深闺人未识”,是作者有意为帝王避讳的说法。}初长成,养在深闺人未识。

天生丽质\footnote{美丽的姿质。}难自弃,一朝选在君王侧。

回眸一笑百媚生,六宫粉黛\footnote{指宫中所有嫔妃。古代皇帝设六宫,正寝(日常处理政务之地)一,燕寝(休息之地)五,合称六宫。粉黛:粉黛本为女性化妆用品,粉以抹脸,黛以描眉。此代指六宫中的女性。无颜色:意谓相形之下,都失去了美好的姿容。}无颜色。

春寒赐浴华清池\footnote{即华清池温泉,在今西安市临潼区南的骊山下。唐贞观十八年(644)建汤泉宫,咸亨二年(671)改名温泉宫,天宝六载(747)扩建后改名华清宫。唐玄宗每年冬、春季都到此居住。},温泉水滑洗凝脂\footnote{形容皮肤白嫩滋润,犹如凝固的脂肪。《经·卫风·硕人》语“肤如凝脂”。}。

侍儿\footnote{宫女。}扶起娇无力,始是新承恩泽\footnote{刚得到皇帝的宠幸。}时。

云鬓\footnote{《木兰诗》:“当窗理云鬓,对镜贴花黄”。形容女子鬓发盛美如云。金步摇:一种金首饰,用金银丝盘成花之形状,上面缀着垂珠之类,插于发鬓,走路时摇曳生姿。}花颜金步摇,芙蓉帐\footnote{绣着莲花的帐子。形容帐之精美。萧纲《戏作谢惠连体十三韵》:珠绳翡翠帷,绮幕芙蓉帐。}暖度春宵\footnote{新婚之夜。}。

春宵苦短日高起,从此君王不早朝。

承欢侍宴无闲暇,春从春游夜专夜。

后宫佳丽三千\footnote{《后汉书·皇后纪》:自武元之后,世增淫费,乃至掖庭三千。言后宫女子之多。据《旧唐书·宦官传》等记载,开元、天宝年间,长安大内、大明、兴庆三宫,皇子十宅院,皇孙百孙院,东都大内、上阳两宫,大率宫女四万人。}人,三千宠爱在一身。

金屋\footnote{《汉武故事》记载,武帝幼时,他姑妈将他抱在膝上,问他要不要她的女儿阿娇作妻子。他笑着回答说:“若得阿娇,当以金屋藏之。”}妆成娇侍夜,玉楼宴罢醉和春。

姊妹弟兄皆列土\footnote{分封土地。据《旧唐书·后妃传》等记载,杨贵妃有姊三人,玄宗并封国夫人之号。长日大姨,封韩国夫人。三姨,封虢国夫人。八姨,封秦国夫人。妃父玄琰,累赠太尉、齐国公。母封凉国夫人。叔玄珪,为光禄卿。再从兄铦,为鸿胪卿。锜,为侍御史,尚武惠妃女太华公主。从祖兄国忠,为右丞相。姊妹,姐妹。},可怜\footnote{可爱,值得羡慕。}光彩生门户。

遂令天下父母心,不重生男重生女\footnote{陈鸿《长恨歌传》云,当时民谣有“生女勿悲酸,生男勿喜欢”,“男不封侯女作妃,看女却为门上楣”等。}。

骊宫\footnote{骊山华清宫。骊山在今陕西临潼。}高处入青云,仙乐风飘处处闻。

缓歌慢舞凝丝竹\footnote{指弦乐器和管乐器伴奏出舒缓的旋律。},尽日君王看不足。

\section{六军不发无奈何}

渔阳\footnote{郡名,辖今北京市平谷县和天津市的蓟县等地,当时属于平卢、范阳、河东三镇节度史安禄山的辖区。天宝十四载(755)冬,安禄山在范阳起兵叛乱。鼙鼓:古代骑兵用的小鼓,此借指战争。}鼙鼓动地来,惊破霓裳羽衣曲\footnote{舞曲名,据说为唐开元年间西凉节度使杨敬述所献,经唐玄宗润色并制作歌词,改用此名。}。

九重城阙\footnote{九重门的京城,此指长安。烟尘生:指发生战事。阙,,意为古代宫殿门前两边的楼,泛指宫殿或帝王的住所。《楚辞·九辩》:君之门以九重。}烟尘生,千乘万骑西南行\footnote{天宝十五载(756)六月,安禄山破潼关,逼近长安。玄宗带领杨贵妃等出延秋门向西南方向逃走。当时随行护卫并不多,“千乘万骑”是夸大之词。乘:一人一骑为一乘。}。

翠华\footnote{用翠鸟羽毛装饰的旗帜,皇帝仪仗队用。司马相如《上林赋》:建翠华之旗,树灵鼍之鼓。百余里:指到了距长安一百多里的马嵬坡。}摇摇行复止,西出都门百余里。

六军\footnote{指天子军队。《周礼·夏官·司马》:王六军。据新旧《唐书·玄宗纪》、《资治通鉴》等记载:天宝十五载(756)六月,哥舒翰至潼关,为其帐下火拔归仁执之降安禄山,潼关不守,京师大骇。玄宗谋幸蜀,乃下诏亲征,仗下后,士庶恐骇。乙未日凌晨,玄宗自延秋门出逃,扈从唯宰相杨国忠、韦见素,内侍高力士及太子、亲王、妃主,皇孙已下多从之不及。丙辰日,次马嵬驿(在兴平县北,今属陕西),诸军不进。龙武大将军陈玄礼奏:逆胡指阙,以诛国忠为名,然中外群情,不无嫌怨。今国步艰阻,乘舆震荡,陛下宜徇群情,为社稷大计,国忠之徒,可置之于法。会吐蕃使遮国忠告诉于驿门,众呼曰:杨国忠连蕃人谋逆!兵士围驿四合,及诛杨国忠、魏方进一族,兵犹未解。玄宗令高力士诘之,回奏曰:诸将既诛国忠,以贵妃在宫,人情恐惧。玄宗即命力士赐贵妃自尽。}不发无奈何,宛转\footnote{形容美人临死前哀怨缠绵的样子。蛾眉:古代美女的代称,此指杨贵妃。《诗经·卫风·硕人》:螓首蛾眉。}蛾眉马前死。

花钿\footnote{用金翠珠宝等制成的花朵形首饰。委地:丢弃在地上。}委地无人收,翠翘\footnote{首饰,形如翡翠鸟尾。金雀:金雀钗,钗形似凤(古称朱雀)。玉搔头:玉簪。《西京杂记》卷二:武帝过李夫人,就取玉簪搔头。自此后宫人搔头皆用玉。}金雀玉搔头。

君王掩面救不得,回看血泪相和流。

黄埃散漫风萧索,云栈萦纡登剑阁\footnote{高入云霄的栈道。萦纡(yíngyū):萦回盘绕。剑阁:又称剑门关,在今四川剑阁县北,是由秦入蜀的要道。}。

峨嵋山\footnote{在今四川峨眉山市。玄宗奔蜀途中,并未经过峨嵋山,这里泛指蜀中高山。}下少人行,旌旗无光日色薄。

蜀江水碧蜀山青,圣主朝朝暮暮情。

行宫\footnote{皇帝离京出行在外的临时住所。}见月伤心色,夜雨闻铃\footnote{《明皇杂录·补遗》:“明皇既幸蜀,西南行。初入斜谷,霖雨涉旬,于栈道雨中闻铃,音与山相应。上既悼念贵妃,采其声为《雨霖铃》曲,以寄恨焉。}肠断声。

天旋地转回龙驭\footnote{指时局好转。肃宗至德二年(757),郭子仪军收复长安。回龙驭:皇帝的车驾归来。},到此踌躇不能去。

马嵬坡下泥土中,不见玉颜空死处\footnote{据《旧唐书·后妃传》载:玄宗自蜀还,令中使祭奠杨贵妃,密令改葬于他所。初瘗时,以紫褥裹之,肌肤已坏,而香囊仍在,内官以献,上皇视之凄惋,乃令图其形于别殿,朝夕视焉。}。

君臣相顾尽沾衣,东望都门信马\footnote{意思是无心鞭马,任马前进。}归。

归来池苑皆依旧,太液\footnote{汉宫中有太液池。未央:汉有未央宫。此皆借指唐长安皇宫。}芙蓉未央柳。

芙蓉如面柳如眉,对此如何不泪垂。

春风桃李花开日,秋雨梧桐叶落时。

西宫南内多秋草,落叶满阶红不扫。

梨园弟子\footnote{指玄宗当年训练的乐工舞女。梨园:据《新唐书·礼乐志》:唐玄宗时宫中教习音乐的机构,曾选"坐部伎"三百人教练歌舞,随时应诏表演,号称“皇帝梨园弟子”。}白发新,椒房\footnote{后妃居住之所,因以花椒和泥抹墙,故称。阿监:宫中的侍从女官。青娥:年轻的宫女。据《新唐书·百官志》,内官宫正有阿监、副监,视七品。}阿监青娥老。

夕殿萤飞思悄然,孤灯挑尽\footnote{古时用油灯照明,为使灯火明亮,过了一会儿就要把浸在油中的灯草往前挑一点。挑尽,说明夜已深。按,唐时宫延夜间燃烛而不点油灯,此处旨在形容玄宗晚年生活环境的凄苦。}未成眠。

迟迟\footnote{迟缓。报更钟鼓声起止原有定时,这里用以形容玄宗长夜难眠时的心情。}钟鼓初长夜,耿耿星河欲曙天\footnote{微明的样子。欲曙天:长夜将晓之时。}。

鸳鸯瓦冷霜华重\footnote{屋顶上俯仰相对合在一起的瓦。《三国志·魏书·方技传》载:文帝梦殿屋两瓦堕地,化为双鸳鸯。房瓦一俯一仰相合,称阴阳瓦,亦称鸳鸯瓦。霜华:霜花。},翡翠衾寒谁与共\footnote{布面绣有翡翠鸟的被子。《楚辞·招魂》:翡翠珠被,烂齐光些。言其珍贵。谁与共:与谁共。}。

悠悠生死别经年,魂魄不曾来入梦。

\section{在天愿为比翼鸟}

临邛道士鸿都客\footnote{意谓有个从临邛来长安的道士。临邛:今四川邛崃县。鸿都:东汉都城洛阳的宫门。},能以精诚致魂魄\footnote{招来杨贵妃的亡魂。}。

为感君王辗转思,遂教方士\footnote{有法术的人。这里指道士。殷勤:尽力。}殷勤觅。

排空驭气\footnote{即腾云驾雾。}奔如电,升天入地求之遍。

上穷碧落下黄泉\footnote{穷尽,找遍。碧落:即天空。黄泉:指地下。},两处茫茫皆不见。

忽闻海上有仙山,山在虚无缥渺间。

楼阁玲珑五云\footnote{华美精巧。五云:五彩云霞。}起,其中绰约\footnote{体态轻盈柔美。《庄子·逍遥游》:藐姑射之山,有神人居焉,肌肤若冰雪,绰约如处子。}多仙子。

中有一人字太真,雪肤花貌参差\footnote{仿佛,差不多。}是。

金阙\footnote{《太平御览》卷六六。引《大洞玉经》:上清宫门中有两阙,左金阙,右玉阙。西厢:《尔雅·释宫》:室有东西厢日庙。西厢在右。玉扃(jiong):玉门。即玉阙之变文。}西厢叩玉扃,转教小玉报双成\footnote{意谓仙府庭院重重,须经辗转通报。小玉:吴王夫差女。双成:传说中西王母的侍女。这里皆借指杨贵妃在仙山的侍女。}。

闻道汉家天子使,九华帐\footnote{绣饰华美的帐子。九华:重重花饰的图案。言帐之精美。《宋书·后妃传》:自汉氏昭阳之轮奂,魏室九华之照耀。}里梦魂惊。

揽衣推枕起徘徊,珠箔\footnote{珠帘。银屏:饰银的屏风。逦迤:接连不断地。}银屏迤逦开。

云鬓半偏新睡觉\footnote{刚睡醒。觉,醒。},花冠不整下堂来。

风吹仙袂\footnote{衣袖。}飘飖举,犹似霓裳羽衣舞。

玉容寂寞\footnote{此指神色黯淡凄楚。阑干:纵横交错的样子。这里形容泪痕满面。}泪阑干,梨花一枝春带雨。

含情凝睇\footnote{凝视。}谢君王,一别音容两渺茫。

昭阳殿\footnote{汉成帝宠妃赵飞燕的寝宫。此借指杨贵妃住过的宫殿。}里恩爱绝,蓬莱宫\footnote{传说中的海上仙山。这里指贵妃在仙山的居所。}中日月长。

回头下望人寰\footnote{人间。}处,不见长安见尘雾。

惟将旧物\footnote{指生前与玄宗定情的信物。}表深情,钿合金钗寄将去\footnote{托道士带回。}。

钗留一股合一扇,钗擘黄金合分钿。

但教心似金钿坚,天上人间会相见。

临别殷勤重寄词\footnote{贵妃在告别是重又托他捎话。},词中有誓两心知\footnote{只有玄宗、贵妃二人心里明白。}。

七月七日长生殿\footnote{在骊山华清宫内,天宝元年(742)造。按“七月”以下六句为作者虚拟之词。陈寅恪在《元白诗笺证稿·长恨歌》中云:“长生殿七夕私誓之为后来增饰之物语,并非当时真确之事实”。“玄宗临幸温汤必在冬季、春初寒冷之时节。今详检两唐书玄宗记无一次于夏日炎暑时幸骊山。”而所谓长生殿者,亦非华清宫之长生殿,而是长安皇宫寝殿之习称。},夜半无人私语时。

在天愿作比翼鸟\footnote{传说中的鸟名,据说只有一目一翼,雌雄并在一起才能飞。},在地愿为连理枝\footnote{两株树木树干相抱。古人常用此二物比喻情侣相爱、永不分离。}。

天长地久有时尽,此恨\footnote{遗憾。绵绵:连绵不断。}绵绵无绝期。

\end{document} 